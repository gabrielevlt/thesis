\section{Soluzione proposta}
\label{sec:soluzione}

\subsection{Architettura del sistema}
\subsubsection{Stack software}
\subsubsection{Comunicazione intermodulo}
\subsubsection{Hardware utilizzato}

\subsection{Modellazione del sistema}
\subsubsection{Funzionalità e casi d'uso}

\subsection{Organizzazione dei dati}
\subsubsection{DBMS}
\paragraph{Progettazione concettuale}
\paragraph{Progettazione logica}
\paragraph{Progettazione fisica}
\paragraph{Adattamento e realizzazione}
\subsubsection{Dati semistrutturati per i modelli di estrazione}
\paragraph{Tecnologia XML}
\paragraph{XML Schema Definition e validazione}
\paragraph{Manipolazione e interrogazione dei dati}

\subsection{Progettazione del sistema}
\subsubsection{Comportamento del software}

\subsection{Implementazione reale}
\subsubsection{Moduli software}
\paragraph{Driver}
\paragraph{Middleware}
\paragraph{Applicativo e GUI}
\subsubsection{Limiti fisici e operativi}

\subsection{Dislocamento}
\subsubsection{Installazione}
\subsubsection{Configurazione}
\subsubsection{Avvio del servizio}
\subsubsection{Discriminazione dell'hardware}

\newpage
\subsection{Rilevamento e risoluzione degli errori}
\subsubsection{Logging degli eventi}
I moduli di questo sistema software dispongono di meccanismi di notifica degli
eventi che si occupano di registrare le principali operazioni effettuate e di
avvertire l'utente al verificarsi di problemi che ne potrebbero impedire il
normale funzionamento.

Il \textit{Log} implementato utilizza, ad eccezione di \textit{Driver}, la
classe \texttt{Logger} del package \textit{java.util.logging} \cite{jianlog} che
oltre a fornire il servizio di logging permette anche di definirne la
\textit{severità}, scegliendo una tra le costanti messe a disposizione dalla
classe \texttt{Level}. Si riportano \citesite{jdoclevel} i livelli \texttt{INFO}
per indicare eventi generici di interesse che non presuppongono un errore,
\texttt{WARNING} da usare in caso si verifichino situazioni che si separano dal
normale flusso di esecuzione e necessitano di essere gestite, \texttt{SEVERE}
per gravi problemi che potrebbero ostacolare il normale funzionamento,
\texttt{OFF} che disabilita totalmente i messaggi di log e \texttt{ALL} per
consentire ad ogni evento di essere registrato indipendentemente dal suo livello.

Data un'istanza di \texttt{Logger} definita o recuperata con il metodo statico
\texttt{getLogger()} sono processati solo i messaggi il cui livello di
severità è maggiore o uguale a quello definito per l'istanza stessa con il
metodo \texttt{setLevel()}.

Severità e testo del messaggio dunque sono parametri da indicare per ogni
\texttt{LogRecord} quando di registra un evento con il metodo \texttt{log()};
i metodi alternativi \texttt{info()}, \texttt{warning()}, \texttt{severe()} non
necessitano del livello di severità.

\vspace{5mm}

Il servizio di logging per il modulo \textit{Driver} è implementato con la
procedura \texttt{logger()} definita come in listato \ref{lst:loggerdef} e
invocata dai thread in esecuzione per stampare i messaggi sul canale
\textit{stdin}.

\vspace{5mm}

\lstinputlisting[caption={Definizione procedura logger().},
label={lst:loggerdef}]{./code/logger.c}

Diversamente dal package di Java, il linguaggio C non dispone dei livelli di
severità. Questa caratteristica è simulata con la variabile globale
\texttt{tolog} inizializzata dal processo padre e acceduta in lettura dai thread
figli così da condizionare il comportamento della procedura. I valori possibili
per \texttt{tolog} sono: \texttt{0} per disattivare il servizio di log,
\texttt{1} per attivarlo.

\vspace{5mm}
 
Il logging degli eventi è inattivo di default tutti i moduli. La sua attivazione
richiede di specificare l'opzione \texttt{-d} al momento dell'esecuzione del software.
\subsubsection{Aggregatore}
bla
\newpage