\section{Soluzione proposta}
\label{sec:soluzione}

\subsection{Architettura del sistema}
\subsubsection{Stack software}
\subsubsection{Comunicazione intermodulo}
\subsubsection{Hardware utilizzato}

\subsection{Modellazione del sistema}
\subsubsection{Funzionalità e casi d'uso}

\subsection{Organizzazione dei dati}
\subsubsection{DBMS}
\paragraph{Progettazione concettuale}
\paragraph{Progettazione logica}
\paragraph{Progettazione fisica}
\paragraph{Adattamento e realizzazione}
\subsubsection{Dati semistrutturati per i modelli di estrazione}
\paragraph{Tecnologia XML}
\paragraph{XML Schema Definition e validazione}
\paragraph{Manipolazione e interrogazione dei dati}

\subsection{Progettazione del sistema}
\subsubsection{Comportamento del software}

\subsection{Implementazione reale}
\subsubsection{Moduli software}
\paragraph{Driver}
\paragraph{Middleware}
\paragraph{Applicativo e GUI}
\subsubsection{Limiti fisici e operativi}

\subsection{Dislocamento}
\subsubsection{Installazione}
\subsubsection{Configurazione}
\subsubsection{Avvio del servizio}
\subsubsection{Discriminazione dell'hardware}

\subsection{Rilevamento e risoluzione degli errori}
\subsubsection{Logging degli eventi}
I moduli di questo sistema software dispongono di meccanismi di notifica degli
eventi che si occupano di registrare le principali operazioni effettuate e di
avvertire l'attore al verificarsi di problemi che ne potrebbero impedire il
normale funzionamento.

Il \textit{Log} implementato utilizza la classe \texttt{Logger} del package
\textit{java.util.logging} \cite{jianlog} che oltre a fornire il servizio
permette anche di definirne la \textit{severità}, scegliendo
una tra le costanti messe a disposizione dalla classe \texttt{Level}. Si riportano
\cite{jdoclevel} i livelli \texttt{INFO} per indicare eventi generici di
interesse che non presuppongono un errore, \texttt{WARNING} da usare in caso si verifichino
situazioni che si separano dal normale flusso di esecuzione e necessitano di
essere gestite, \texttt{SEVERE} per problemi gravi che potrebbero impedire il
normale funzionamento, \texttt{OFF} che disabilita totalmente i messaggi di log
e \texttt{ALL} per affermare che ogni evento deve essere registrato
indipendentemente dal suo livello.

Data un'istanza di \texttt{Logger} definita o recuperata con il metodo statico
\texttt{getLogger()} saranno processati solo i messaggi il cui livello di
severità è maggiore o uguale a quello definito per l'istanza stessa con il
metodo \texttt{setLevel()}.

Severità e testo del messaggio dunque sono parametri da indicare per ogni
\texttt{LogRecord}; per presentarlo si ha a disposizione il metodo
\texttt{log()} oppure uno tra \texttt{info()}, \texttt{warning()},
\texttt{severe()}: in questo caso non è necessario indicare il livello di
severità.
\subsubsection{Aggregatore}